
\subsubsection{Memory hierarchy}
\label{sec:memhier}

For replication purposes, it is important to thoroughly characterize
the machine we are using to run the tests. All experiments were
carried out on a dual-processor 6-core Intel Xeon CPU (E5645) running
at 2.40GHz, for a total of 12 physical cores at
28.8GHz. Hyperthreading was disabled. Figure \ref{fig:topology} shows
the memory hierarchy of the testing computer. To save space, we only
show one of the chips (Socket P\#0), the other is identical.

% \begin{figure}[htp]
%   \centering
%   \includegraphics[scale=.5]{topology.pdf}
%   \caption{The memory hierarchy of the testing computer}
%   \label{fig:topology}
% \end{figure}

Each core has one processor unit (PU), with separate L1d (32KB) and L2
(256KB) caches. (The strange numbers of Cores and PUs have to do with
how the Linux Kernel sees them logically.) They share a 12MB L3
cache. Main memory is 6GB. The computer runs Linux 3.0.0-15-server, in
64-bit mode. The details of the caches are as follows:

\small
\begin{verbatim}
  L1 Data:        32K 8-way with 64 byte lines
  L2 Unified:     256K 8-way with 64 byte lines
  L3 Unified:     12288K 16-way with 64 byte lines
\end{verbatim}
\normalsize

%%% Local Variables: 
%%% mode: latex
%%% TeX-master: "sat"
%%% End: 
