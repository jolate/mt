\subsection{SAT and SAT solvers}

Let $v \in V $ be a \emph{propositional variable} and $V$
 a fixed finite set of propositional symbols.  If $v \in V$,
then $v$ and $\lnot v$ are \emph{literals} of $V$.  
The \emph{negation} of a literal $l$, written $\lnot l$, denotes 
$\lnot v$ if $l$ is $v$, and $v$ if $l$ is $\lnot v$.
A \emph{clause} is a disjunction of literals $l_1 \lor\ldots\lor l_n$.
A (CNF) \emph{formula} is a conjunction of one or
more clauses $C_1 \land\ldots\land C_n$. 
 A (partial truth) \emph{assignment} $M$ is a
set of literals such that $\{ v, \lnot v \} \subseteq M$ for no $v$. A
literal $l$ is \emph{true} in $M$ if $l \in M$, is \emph{false} in $M$
if $\lnot l \in M$, and is \emph{undefined} in $M$ otherwise. 
A clause $C$ is true in $M$ if at least one of its
literals is true in $M$.  It is false in $M$ if all its literals are
false in $M$, and it is undefined in $M$ otherwise. 
A formula $F$ is true in $M$, or
\emph{satisfied} by $M$, if all its clauses are
true in $M$.  In that case, $M$ is a \emph{model} of $F$.  If $F$ has
no models then it is \emph{unsatisfiable}.  

The problem we are interested in is the \emph{SAT problem}: given a
formula $F$, decide whether there exists a model of $F$ or not.
Since there exists a polynomial transformation
(see~\cite{Tseitin1968}) for any arbitrary formula to an
equisatisfiable CNF one, we will assume w.l.o.g. that $F$ is in CNF.

A software program that solves this problem is called a \emph{SAT
  solver} The Conflict-Driven-Clause-Learning (CDCL) algorithm, is
nowadays at the basis of most state-of-ther-art
SAT-solvers\cite{gluclose,plingeling,cryptominisat}.  This algorithm
has, at its roots, the very simple DPLL
algorithm~\cite{Davisetal1962CACM}. Thanks to the work done, mainly,
in
\cite{relsat,Chaff2001,GRASP1999IEEE,ZhangStickel1996AIMATH,EenSorensson2003SAT,picosat2008}
it has evolved to an algorithm that incorpores several conceptual and
implementation improvements, making modern SAT-solvers able to handle
formulas of millons of variables and clauses. Algorithm \ref{alg:cdcl}
sketches the CDCL algorithm.

\begin{algorithm}[h]
\SetAlgoLined
\SetKwData{Conflict}{conflict}
\SetKwData{Dl}{dl}
\SetKwData{Lemma}{lemma}
\SetKwData{Dec}{dec}
\SetKwData{Status}{status}
\SetKwData{Model}{model}
\SetKwData{Result}{res}\SetKwData{BV}{BV}
\SetKwInOut{Input}{Input}\SetKwInOut{Output}{Output}
\Input{formula $F=\{C_1,\ldots,C_m\}$}
\Output{SAT OR UNSAT}
 \BlankLine
 \Status \assign UNDEF\;
 \Model \assign \{\}\;
 \Dl \assign $0$\;
 \While{\Status == UNDEF}{
   (\Conflict, \Model) \assign BCP(\Model, $F$)\;
   \While{ \Conflict $\neq$ NULL}{
     \If{\Dl == 0}{
       \Return UNSAT\;
     }
     \Lemma \assign CONFLICT\_ANALYSIS(\Conflict, \Model, $F$)\;
%     \Lemma \assign LEMMA\_SHORTENING(\Lemma,\Model,$F$)\;
     $F$ \assign F $\land$ \Lemma\;
     \Dl \assign LARGEST\_DL(\Lemma, \Model)\;
     \Model \assign BACKJUMP\_TO\_DL(\Dl, \Model)\;
     (\Conflict, \Model) \assign BCP(\Model, $F$)\;
   }

   \If{\Status == UNDEF }{
     \Dec \assign DECIDE(\Model, $F$)\;
     \Dl \assign \Dl +1\;
     \lIf{ \Dec = $0$ }{\Status \assign SAT\;}
     \Model \assign \Model $\cup$ \{ \Dec\}\;
   }

 }
 \Return \Status
 \caption{CDCL algorithm}
 \label{alg:cdcl}
\end{algorithm}

Basically, the CDCL algorithm is a backumping search algorithm that
incrementally builds a partial assignment $M$ over iterations of the
\emph{DECIDE} and \emph{Binary Constraint Propagation (BCP)} procedures, 
returning SAT if
$M$ becomes a model of $F$ or UNSAT if no such model exists.

The DECIDE procedure correspond to a branching step of the search and
applies when the unit propagation procedure can not set true any
further literal (see below).  When no inference can be done about
which literals should be true in $M$, a literal $l^{dl}$ is chosen
(guessed) and added to $M$ in order to continue the search.  Each time
a new decision literal $l^{dl}$ is added to $M$ the \emph{ decision
  level} $dl$ of the search is increased and we say that all the
literals in $M$ after $l^{dl}$ and before $l^{dl+1}$ belong to
decision level $dl$.  For further reading about the DECIDE procedure,
we refer to \cite{Chaff2001,MiniSat,rsat}.

The BCP procedure applies when certain assignment $M$
falsifies all literals but one, of a clause $C$.  If $l$ is the
undefined literal in $C$, in order for $M$ to become a model of $F$,
$l$ must be added to $M$ for $C$ to be satisfied. This procedure ends
when there is no literal left to add to $M$ or when it finds that all
literals of certain clause $C$ are false.  If it happens the first
case, UP returns an updated model containing all such propagations and
the search continues.  In the second case it returns the falsified
clause which we call a \emph{conflicting clause}.  For details on how
modern SAT Solvers implement this procedure and literature regarding
Cache performance, mainly related with UP, we refer to
\cite{Chaff2001,ZhangMalik2003SAT}
 
If unit propagation finds a conflicting clause, then two possibilities
apply. If the decision level is zero
(i.e. no decisions have been made) then the CDCL procedure returns
UNSAT. If this is not the case, a \emph{CONFLICT ANALISIS} procedure
is called. This procedure analizes the cause of such conflict
(i.e. determines which decisions have driven to this conflict) and
returns a new clause (which we call a \emph{lemma}) that is entailed
by the original formula. Then, the algorithm backjumps to an earlier
decision level $dl'$ that corresponds to the highest $dl'<dl$ in the
lemma, and propagates with it.  CONFLICT ANALISIS works in such a way
that, when backjumping and propagating with the lemma, the original
conflict is avoided. The lemmas learned at CONFLICT ANALISIS time are
usually added to the formula in order to avoid similar conflicts and
can also be deleted (when the formula is too big and they are no
longer needed). For details of this procedure, we refer to
\cite{GRASP1999IEEE,Zhang2001} and for lemma deletion heuristics to
\cite{Relsat97,Goldberg2002DATE,gluclose}.

For a complete review of this algorithm as well as proofs over its
termination and soundness we refer to \cite{Nieuwenhuisetal2006JACM}


\subsection{Parallel SAT solvers}

Parallel SAT solvers are not as mature as sequential ones and it is
still not clear which path to follow when designing and implementing
such new solvers. In this section we will briefly present the two main approaches
used in parallel SAT solvers for shared memory architectures. The revision on 
SAT solving in distributed memory architectures are out of the scope of this paper.
We mainly classify the parallel SAT solvers for shared memory architectures into two
categories: Portfolio approach solvers and search space splitting ones.

The main idea behind portfolio approach solvers is the fact that
different strategies/parameters of CDCL sequential solvers or even 
different kind of sequential solvers
perform better for different families of SAT problems.  In sequential
CDCL SAT-solving, for example, there exist several parameters/strategies related
with the algorithm's heuristics in restarting, deciding
or cleaning the clause database. Taking this into consideration, a
portfolio approach is very straightforward: Run a group of sequential
solvers in different threads, each with different parameters and/or
different strategies. This idea can be easily extrapolated to other
solution beyond classic CDCL as well. 
The time the portfolio-approach based solver
will take to solve the problem will be the time of the fastest thread
in the group of solvers running in parallel.  Differences between this
kind of solvers lie in whether the clause database should be physically
shared~\cite{Sartagnan} or, otherwise, if each thread should have its
own database. If this second approach is taken, it is possible to implement 
the solver so that its different threads interchange lemmas 
both: aggressively~\cite{ManySAT} or
selectively~\cite{plingeling} or avoiding communications between
threads at all~\cite{ppfolio}.

Search space splitting solvers do not run different solvers in
parallel, they run one solving instance that splits the search space 
into disjoint subspaces. A common strategy
to divide the search space is to use guiding paths~\cite{psato}. A
guiding path is a partial assignment $M$ in $F$, which restricts the
search space of the SAT problem. A solver that divides its search
space with guiding paths will assign threads to solve $F$ with the
given $M$ from the guiding path the thread was assigned to. Once a
thread finishes searching a guiding path with no success, it can
request another to keep searching (we refer to \cite{paMiraXT} for
further explanations).

Both parallelization strategies (portfolio approach and
search-space-splitting approach) were and are currently being applied to
shared memory parallel computers (e.g. \cite{plingeling}) as well as
distributed memory ones~(e.g. \cite{paMiraXT}). For a further review
on shared memory parallel sat solving, we refer to \cite{survey-psolvers}

In what follows, we only work with solvers implementing the portfolio approach
since this is the currently dominating approach and the one that best
results has reported in recent papers and competitions.

